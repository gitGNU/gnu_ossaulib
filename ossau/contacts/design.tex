\documentclass{article}

\begin{document}

\section{Introduction}

A common modern experience is to have contact information in many
different places.  In Outlook, BBDB, Evolution, your mobile phone,
your PDA, and so on.\footnote{And of course, in a paper address book
  --- but unfortunately software cannot help with that.}  This article
discusses the problem of keeping this information in sync, and
proposes a solution.

Keeping contact information ``in sync'' means copying changes that are
made in one place --- new contacts, removal of contacts that are no
longer wanted, and updates to existing contacts --- to all the other
places where those changes make sense.  Conceptually this is simple,
but complications start to arise when one starts looking at the
details.

\begin{itemize}

\item When different changes are made to a contact in different
  places, how can those changes easily be reconciled?  (Bearing in
  mind that the different places may use different formats for their
  contact information.)

\item If the contact database from one place has a particular contact,
  and that from another place does not, how can we tell if that
  contact should exist in the merged database?  (It might have been
  deliberately deleted from the second place's contacts, and therefore
  should be deleted from other places too.)

\item Suppose a contact database in one place can contain names,
  addresses and phone numbers, while that in another place can have
  names and phone numbers only.  When incorporating updated and new
  contacts from the latter place, how can the software avoid thinking
  that all of the address information has been deliberately deleted
  (and so delete it from the merged database too)?

\end{itemize}

As far as I know there isn't yet any free software that addresses
these problems.  The rest of this article is my proposal for a program
that does.

\section{Ideas}

It seems to me that the processing required for contact
synchronisation is a combination of two things.

\begin{itemize}

\item Conversion between different contact information formats.

\item Revision history tracking.

\end{itemize}

The first of these is obvious.  The second is less so but actually
crucial, because when we have two versions $X$ and $Y$ of a contact,
and are trying to decide which is better, history allows a program to
see how $X$ and $Y$ each differ from the previously known information
for that contact; in practice this often means that the program can
reliably and automatically determine what the merged contact should
be.  Maintaining a history also means that it is not a disaster if the
program occasionally makes a mistake, because the user can retrieve
any lost information.

There is already excellent free software that does revision history
tracking, namely Git.\footnote{And of course there are many other
  candidates, but Git is the one that I am most familiar with.}  If we
choose to use Git to store and manage the repository of current and
historical contacts, the other points that we need to specify, for a
complete design, are as follows.

\begin{itemize}

\item The format of the master database for contact information.

\item In general, how conversion routines for particular other formats
  will convert to and from this master format.

\item How the high level operations of direct interest to the user
  will map onto operations on the Git repository.

\end{itemize}

\section{Master database format}

In the master database, each contact is in its own file, named
``\_FIRST-NAMES\_LAST-NAME''.  The contents of a contact file look like
this:

\begin{verbatim}
FIRST-NAMES
 John Henry
LAST-NAME
 Doe
ADDRESS home
 45 Church Street
 London
 E8 4JK
ADDRESS work
 8 Russell Square
 Portsmouth
 PO21 8HH
PHONE home
 020 7733 9245
X-BBDB-NOTES
 Miscellaneous notes here
\end{verbatim}

The key design points of this format are that

\begin{itemize}

\item it is easily readable and mergeable by humans.  Most merging
  tools work line-by-line, so it is helpful for merging to have each
  piece of information on its own line.

\item it supports multiple phone numbers and addresses

\item it supports additional fields that are specific to particular
  contact systems (in this case BBDB) and cannot be converted to other
  systems, but still need to be preserved and tracked.

\end{itemize}

Technical details are as follows.

\begin{itemize}

\item The contact is a collection of fields.

\item Each field consists of
  \begin{itemize}
    \item the field name, in capitals
    \item in some cases (ADDRESS and PHONE), a descriptive identifier
      for the field, to distinguish it from other instances of the
      same field
    \item the field data, on a series of lines beginning with a space.
  \end{itemize}

\item The fields may be arranged in any order.

\end{itemize}

\section{Conversion to and from the master format}

[...to be continued...]

\section{Mapping between standard fields and native fields}

Assume an entry of each supported format can be mapped to an alist
such that

\begin{enumerate)

\item it can be formulaically written out in the canonical contact
  format

\item it can be read back in from the canonical contact format, and
  will produce an identical record in the source format

\end{enumerate}

Thus far, every format has a distinct and completely non-overlapping
set of alist key names.  e.g. X-BBDB-..., X-GOOGLE-...

Now we define how format-specific key values are mapped to common key
values.

Any format-specific key values that can be exactly reconstructed from
common key values can then be discarded.

\end{document}
